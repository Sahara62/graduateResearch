\documentclass{tpu-sotu}
\usepackage{ascmac}
\usepackage{cite}
\usepackage[dvipdfmx]{hyperref}
\usepackage{pxjahyper}
\usepackage{listings}%begin{lstlisting}で使える
\lstset{
basicstyle={\small},% 
identifierstyle={\small},% 
commentstyle={\small\ttfamily \color[rgb]{0,0,0}},% 
keywordstyle={\small\bfseries \color[rgb]{0,0,1}},% 
ndkeywordstyle={\small},% 
stringstyle={\small\ttfamily}, 
frame={tb}, 
breaklines=true, 
columns=[l]{fullflexible},% 
numbers=left,%これ消すと行番号消える
xrightmargin=0zw,% 
xleftmargin=3zw,% 
numberstyle={\scriptsize},% 
stepnumber=1, 
numbersep=1zw,% 
morecomment=[l]{//}% 
}

\title{UPPAALを用いた自動運転車の\\群制御アルゴリズムのモデル化と検証}
\etitle{title}
\author{佐原優衣}
\gakusekibangou{1415045}
\date{2018年2月}
\professor{中村 正樹 准教授}
\department{電子・情報工学科}
%- - - - - - - - - - - - - - - - - - - - - - - - -
\begin{document}
\maketitle
\clearpage
\pagenumbering{roman}
\setcounter{tocdepth}{3}
\tableofcontents
\clearpage
\pagenumbering{arabic}

\chapter{はじめに}
	\section{背景}
	移動手段として自動車はよく使われる。自動車の交通整備の方法として信号機が広く採用されている。しかし,交通量が少ない場所では信号機は導入されていないこともある。人が運転する場合は周囲に他車がいないこと,歩行者などがいないことを確認して通過する。では,自動運転車の場合を考える。
		\subsection{問題提起}
		街の中で多数の自動運転車が他車を考慮しない経路選択をすると問題が発生する恐れがある。そのためには個々による車の制御だけでなく群制御則を導入する必要があるだろう
	\section{目的}
	自動運転車の群制御アルゴリズムを形式的に記述し,モデル検査を用いて,性質を検証する。
	\section{論文の構成}
	本論文では,
\chapter{アプローチ}

	\section{街の規格と車の規格}
	\section{問題提起}
	\section{モデル検査ツールUPPAAL}
	本節では文献\cite{a1}からモデル検査ツールUPPAALの概説を行う。
	\subsection{モデル検査と時間の扱い}
	モデル検査は,システム上で起こり得る状態を網羅的に調べることにより設計の誤りを発見する自動検証手法の一種である。モデル検査手法は,システムの振る舞いの設計,および検証したい性質をそれぞれモデル化し,ツールを用いて,システムが性質を満たしているかを調べる。
\chapter{UPPAALによる\\モデル化と検証}
	\section{群制御アルゴリズム}
	\section{街のモデル化}
\chapter{考察}
\chapter{おわりに}
\acknowledgements
\begin{thebibliography}{1}
	\bibitem{a1}{長谷川哲夫,田原康之,磯部祥尚,UPPAALによる性能モデル検証ーリアルタイムシステムのモデル化と検証ー,(株)近代科学社,2012.}
\end{thebibliography}
\end{document}